
%%%%%%%%%%%% Abstract template for AUS-ICMS15.  %%%%%%%%%%%%%%%%


%%%%%%%%%%%% Please do NOT modify the preamble. %%%%%%%%%%%%%

\documentclass[11pt]{article}
\usepackage{amssymb}
\pagestyle{empty}
\textwidth178mm
\textheight265mm
\oddsidemargin-10mm
\topmargin-14mm
\parskip4pt plus2pt minus2pt
\newcommand{\titleA}[1]{\begin{center} \textbf{\large #1}\end{center}}
\newcommand{\authorA}[1]{\begin{center} #1 \end{center}}
\newcommand{\addressA}[2]{\begin{center}  #1 \textit{#2} \end{center} }
\newcommand{\researcharea}[1]{\noindent \textbf{General area of research:} #1  }
\newcommand{\abstractA}[1]{\medskip  \noindent \textbf{Abstract:} #1}
\newcommand{\keywordsA}[1]{\medskip \noindent \textbf{Keywords:} #1}
\usepackage{fancyhdr}%
\usepackage{fancyheadings}
\pagestyle{fancy}%
\begin{document}
{\lhead{\footnotesize{{\tiny\scshape \small{  Third International Conference on Mathematics and Statistics (AUS-ICMS’20) February 6-9, 2020,  Sharjah, UAE }}}}
}
\rhead{}
%%%%% Enter your information below %%%%%%%%%

\titleA{WHEN IS A PUISEUX MONOID ATOMIC?}

\authorA{Scott T. Chapman, Felix Gotti, {\bf Marly Gotti}}
%The presenter's name must be in bold.

\addressA{Department of Mathematics\\
	Sam Houston State University, Huntsville, TX 77341 \\
	{\it scott.chapman@shsu.edu}
	\vspace{1pt}
	
	Department of Mathematics\\
	UC Berkeley, Berkeley, CA 94720 \\
	{\it felixgotti@berkeley.edu}
	\vspace{1pt}
	
	Department of Mathematics\\
	University of Florida, Gainesville, FL 32611 \\
	{\it marlycormar@ufl.edu}
}\\
%\researcharea{ ...... (Please choose from  the list of conference topics if applicable.)}

\abstractA{
		A Puiseux monoid is an additive submonoid of the nonnegative cone of~$\mathbb{Q}$. If $M$ is a Puiseux monoid, then the question of when each nonunit element of $M$ can be written as a sum of irreducible elements (or is \textit{atomic}) is surprisingly difficult.  For instance, although various techniques have been developed over the past few years to identify subclasses of Puiseux monoids which are atomic, no general characterization of such monoids is known. Here we discuss some of the most relevant aspects related to the atomicity of Puiseux monoids. We provide characterizations of when $M$ is finitely generated, factorial, half-factorial, other-half-factorial, Pr\"ufer, seminormal, root-closed, and completely integrally closed. In addition to the atomic property, precise characterizations are also not known for when $M$ satisfies the ACCP, is a BF-monoid, or is an FF-monoid; in each of these four cases, we construct classes of Puiseux monoids satisfying these properties. 

}

\keywordsA{Puiseux monoids, atomicity, factorization theory, Pr\"ufer monoids, numerical monoids, ACCP, BF-monoids, FF-monoids.} \\

\noindent{\bf {\small {2010 Mathematics Subject Classification}: Primary: 20M13; Secondary: 06F05, 20M14}}

\begin{thebibliography}{99}
	
	\bibitem{CGG19} S.~T. Chapman, F. Gotti, and M. Gotti, \emph{Factorization invariants of Puiseux monoids generated by geometric sequences}, Comm. Algebra (2019), https://doi.org/10.1080/00927872.2019.1646269
	
	\bibitem{GGT19} A.~Geroldinger, F. Gotti, and S. Tringali: \emph{On strongly primary monoids, with a focus on Puiseux monoids}. In arXiv: https://arxiv.org/pdf/1910.10270.pdf
	
	\bibitem{fG19} F. Gotti: \emph{Increasing positive monoids of ordered fields are FF-monoids}, J. Algebra \textbf{518} (2019) 40--56.
	
	\bibitem{fG17} F. Gotti: \emph{On the atomic structure of Puiseux monoids}, J. Algebra Appl. \textbf{16} (2017) 1750126.
	
	\bibitem{GG17} F. Gotti and M. Gotti: \emph{Atomicity and boundedness of monotone Puiseux monoids}, Semigroup Forum \textbf{96} (2018) 536--552.
	
	\bibitem{mG19} M. Gotti: \emph{On the local k-elasticities of Puiseux monoids}, Internat. J. Algebra Comput. {\bf 29} (2019) 147--158.
	
\end{thebibliography}

\end{document}

